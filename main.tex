%-------------------------
% Resume in Latex
% Author : Jake Gutierrez
% Based off of: https://github.com/sb2nov/resume
% License : MIT
%------------------------

\documentclass[letterpaper,11pt]{article}

\usepackage{latexsym}
\usepackage[empty]{fullpage}
\usepackage{titlesec}
\usepackage{marvosym}
\usepackage[usenames,dvipsnames]{color}
\usepackage{verbatim}
\usepackage{enumitem}
\usepackage[hidelinks]{hyperref}
\usepackage{fancyhdr}
\usepackage[russian,english]{babel}
\usepackage[utf8]{inputenc}
\usepackage[T2A]{fontenc}
\usepackage{tabularx}
\input{glyphtounicode}


%----------FONT OPTIONS----------
% sans-serif
% \usepackage[sfdefault]{FiraSans}
% \usepackage[sfdefault]{roboto}
% \usepackage[sfdefault]{noto-sans}
% \usepackage[default]{sourcesanspro}

% serif
% \usepackage{CormorantGaramond}
% \usepackage{charter}


\pagestyle{fancy}
\fancyhf{} % clear all header and footer fields
\fancyfoot{}
\renewcommand{\headrulewidth}{0pt}
\renewcommand{\footrulewidth}{0pt}

% Adjust margins
\addtolength{\oddsidemargin}{-0.5in}
\addtolength{\evensidemargin}{-0.5in}
\addtolength{\textwidth}{1in}
\addtolength{\topmargin}{-.5in}
\addtolength{\textheight}{1.0in}

\urlstyle{same}

\raggedbottom
\raggedright
\setlength{\tabcolsep}{0in}

% Sections formatting
\titleformat{\section}{
  \vspace{-4pt}\scshape\raggedright\large
}{}{0em}{}[\color{black}\titlerule \vspace{-5pt}]

% Ensure that generate pdf is machine readable/ATS parsable
\pdfgentounicode=1

%-------------------------
% Custom commands
\newcommand{\resumeItem}[1]{
  \item\small{
    {#1 \vspace{-2pt}}
  }
}

\newcommand{\resumeSubheading}[4]{
  \vspace{-2pt}\item
    \begin{tabular*}{0.97\textwidth}[t]{l@{\extracolsep{\fill}}r}
      \textbf{#1} & #2 \\
      \textit{\small#3} & \textit{\small #4} \\
    \end{tabular*}\vspace{-7pt}
}

\newcommand{\resumeSubSubheading}[2]{
    \item
    \begin{tabular*}{0.97\textwidth}{l@{\extracolsep{\fill}}r}
      \textit{\small#1} & \textit{\small #2} \\
    \end{tabular*}\vspace{-7pt}
}

\newcommand{\resumeProjectHeading}[2]{
    \item
    \begin{tabular*}{0.97\textwidth}{l@{\extracolsep{\fill}}r}
      \small#1 & #2 \\
    \end{tabular*}\vspace{-7pt}
}

\newcommand{\resumeSubItem}[1]{\resumeItem{#1}\vspace{-4pt}}

\renewcommand\labelitemii{$\vcenter{\hbox{\tiny$\bullet$}}$}

\newcommand{\resumeSubHeadingListStart}{\begin{itemize}[leftmargin=0.15in, label={}]}
\newcommand{\resumeSubHeadingListEnd}{\end{itemize}}
\newcommand{\resumeItemListStart}{\begin{itemize}}
\newcommand{\resumeItemListEnd}{\end{itemize}\vspace{-5pt}}

%-------------------------------------------
%%%%%%  RESUME STARTS HERE  %%%%%%%%%%%%%%%%%%%%%%%%%%%%


\begin{document}

%----------HEADING----------
% \begin{tabular*}{\textwidth}{l@{\extracolsep{\fill}}r}
%   \textbf{\href{http://sourabhbajaj.com/}{\Large Sourabh Bajaj}} & Email : \href{mailto:sourabh@sourabhbajaj.com}{sourabh@sourabhbajaj.com}\\
%   \href{http://sourabhbajaj.com/}{http://www.sourabhbajaj.com} & Mobile : +1-123-456-7890 \\
% \end{tabular*}

\begin{center}
    \textbf{\Huge \scshape Сухоруков Даниил Андреевич} \\ \vspace{1pt}
    \small 8-981-441-81-25 $|$ \underline{densuh2014@ya.ru} $|$ 
    \href{https://github.com/Code-oclock?tab=repositories}{\underline{Code-oclock}} $|$
    \href{https://t.me/sir_uwu}{\underline{t.me/sir\_uwu}}
\end{center}


%-----------EDUCATION-----------
\section{Education}
  \resumeSubHeadingListStart
    \resumeSubheading
      {НИУ ИТМО}{Санкт-Петербург}
      {Факультет информационных технологий и программирования}{2023 -- 2027}
  \resumeSubHeadingListEnd

%-----------PROJECTS-----------
\section{Projects}
    \resumeSubHeadingListStart
      \resumeProjectHeading
          {\textbf{Float numbers} $|$ \emph{С}}{Февраль 2023 -- Март 2023}
          \resumeItemListStart
            \resumeItem{Собственная программная реализация чисел с плавающей точкой}
            \resumeItem{Реализована вся арифметика, сравнение и форматирование}
          \resumeItemListEnd
      \resumeProjectHeading
          {\textbf{Single-cycle processor} $|$ \emph{Verilog}}{Ноябрь 2023 -- Декабрь 2023}
          \resumeItemListStart
            \resumeItem{Программная реализация однотактного процессора на транзисторах}
            \resumeItem{Успешная симуляция выполнения тестовых программ (арифметика, ветвления)}
          \resumeItemListEnd
      \resumeProjectHeading
          {\textbf{VPN-bot} $|$ \emph{Python}}{Сентябрь 2024 -- Февраль 2025}
          \resumeItemListStart
            \resumeItem{Реализован Telegram-бот для автоматизированного туннелирования сетевого трафика}
            \resumeItem{Интегрированы платежи}
            \resumeItem{В качестве платежного шлюза используется ЮMoney и CryptoCloud}
          \resumeItemListEnd
      \resumeProjectHeading
          {\textbf{Generic expression parser} $|$ \emph{Java}}{Апрель 2024 -- Май 2024}
          \resumeItemListStart
            \resumeItem{Реализован универсальный парсер выражений}
            \resumeItem{Поддержка любых типов данных и порядка операций}
            \resumeItem{Реализованы операции сложения, умножения, деления, вычитания, унарного минуса}
          \resumeItemListEnd
      \resumeProjectHeading
          {\textbf{LFU-cache} $|$ \emph{Go}}{Октябрь 2024 -- Ноябрь 2024}
          \resumeItemListStart
            \resumeItem{Реализована библиотека-кэш с политикой удаления наименее часто используемых элементов}
            \resumeItem{В библиотеке использовалась собственная реализация linked list}
          \resumeItemListEnd
      \resumeProjectHeading
          {\textbf{Multithreaded Map-reduce crawler} $|$ \emph{Go}}{Ноябрь 2024 -- Декабрь 2024}
          \resumeItemListStart
            \resumeItem{Разработан потокобезопасный пул воркеров (Transform, Accumulate, List) для параллельной обработки файлов}
            \resumeItem{Реализован краулер с рекурсивным обходом файловой системы, поддерживающий конфигурацию числа воркеров (поиск, обработка, агрегация)}
            \resumeItem{Обеспечена отмена операций через контекст}
            \resumeItem{Применено паттерн Map-Reduce: Searcher (BFS обход), Transformer (обработка файлов), Accumulator (агрегация результатов)}
            \resumeItem{Оптимизирована производительность через параллелизм и балансировку нагрузки между горутинами}
          \resumeItemListEnd
      \resumeProjectHeading
          {\textbf{Go Assembly} $|$ \emph{Go}}{Октябрь 2024 -- Ноябрь 2024}
          \resumeItemListStart
            \resumeItem{Реализованы базовые алгоритмы и функции: сумма на слайсе, lowerbound, fibonacci}
            \resumeItem{Реализация основана на amd64}
          \resumeItemListEnd
      \resumeProjectHeading
          {\textbf{Service “Library”} $|$ \emph{Go, PostgreSql}}{Декабрь 2024 -- Апрель 2025}
          \resumeItemListStart
            \resumeItem{Микросервис для управления библиотекой: добавление, поиск, выдача книг/авторов через REST API}
            \resumeItem{Создан при помощи protobuf, grpc, require, gomock, postgresql}
            \resumeItem{Реализация с 100\% покрытием тестов, использующих моки}
            \resumeItem{Хранение осуществляется в бд, созданной при помощи Postgresql}
            \resumeItem{Использованы паттерны WorkerPool и OutBox, реализованы CRUD-эндпоинты и контейнеризация, при помощи Docker}
          \resumeItemListEnd
    \resumeSubHeadingListEnd



%
%-----------PROGRAMMING SKILLS-----------
\section{Technical Skills}
 \begin{itemize}[leftmargin=0.15in, label={}]
    \small{\item{
     \textbf{Languages}{: Go, Java, Python, C/C++, SQL (Postgres), JavaScript, HTML/CSS} \\
     \textbf{Developer Tools}{: Git / GitHab, Docker, Linux / Bash, Postman, VS Code, Visual Studio, PyCharm, IntelliJ} \\
    }}
 \end{itemize}

%-----------SOFT SKILLS-----------
\section{Soft Skills}
 \begin{itemize}[leftmargin=-0.1in, label={}] % Полное удаление отступа
    \resumeItemListStart
        \resumeItem{Английский - B2}
        \resumeItem{Курс коммуникаций и командообразования}
        \resumeItem{Готовность обучаться и брать на себя ответственность}
        \resumeItem{Присутствует небольшой опыт работы в команде}
    \resumeItemListEnd
 \end{itemize}


%-------------------------------------------
\end{document}
